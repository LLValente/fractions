\documentclass{article}

\usepackage{graphicx}
\usepackage[brazilian]{babel}

\title{Avaliação}
\author{Lucas L. Valente}        
\date{2023-06-11}

\begin{document}
\maketitle
\newpage

\begin{enumerate}
	\item Calcule. \textbf{(2.0 pnts)}

		\begin{enumerate}
			\item $\frac{4}{8} + \frac{14}{12}$
			\item $\frac{14}{12} - \frac{2}{3}$
			\item $\frac{2}{3} . \frac{1}{8}$
			\item $\frac{1}{8} : \frac{12}{6}$
			\item $\left(\frac{12}{6}\right) ^ 2$
		\end{enumerate}

	\item Observe as frações em cada item abaixo e assinale $<$, $>$ ou $=$. \textbf{(1.5 pnts)}

		\begin{enumerate}
			\item $\frac{8}{13} e \frac{4}{5}$
			\item $\frac{4}{5} e \frac{6}{14}$
			\item $\frac{6}{14} e \frac{3}{2}$
			\item $\frac{3}{2} e \frac{10}{15}$
			\item $\frac{10}{15} e \frac{11}{12}$
			\item $\frac{11}{12} e \frac{5}{1}$
		\end{enumerate}

	\item Encontre a fração geratriz para cada dízima periódica abaixo. \textbf{(2.0 pnts)}

		\begin{enumerate}
			\item $\frac{8}{13}$
			\item $\frac{2}{5}$
			\item $\frac{4}{12}$
			\item $\frac{4}{10}$
		\end{enumerate}

	\item Escreva por extenso o nome de cada fração abaixo. \textbf{(1.0 pnts)}

		\begin{enumerate}
			\item $\frac{6}{13}$
			\item $\frac{11}{15}$
			\item $\frac{12}{13}$
			\item $\frac{5}{15}$
		\end{enumerate}

	\item Calcule as potências a seguir. \textbf{(1.5 pnts)}

		\begin{enumerate}
			\item $\frac{11}{3}$
			\item $\frac{3}{7}$
			\item $\frac{9}{11}$
			\item $\frac{6}{8}$
			\item $\frac{9}{8}$
		\end{enumerate}

	\item Para cada fração a seguir, faça uma representação geométrica. \textbf{(1.0 pnts)}

		\begin{enumerate}
			\item $\frac{8}{2}$
			\item $\frac{1}{11}$
			\item $\frac{4}{5}$
			\item $\frac{14}{6}$
			\item $\frac{12}{1}$
			\item $\frac{5}{10}$
		\end{enumerate}

\end{enumerate}
\end{document}