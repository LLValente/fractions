\documentclass{article}

\usepackage{graphicx}
\usepackage[brazilian]{babel}

\title{Avaliação}
\author{Lucas L. Valente}        
\date{2023-06-11}

\begin{document}
\maketitle
\newpage

\begin{enumerate}
	\item Calcule. \textbf{(2.0 pnts)}

		\begin{enumerate}
			\item $\frac{9}{12} + \frac{7}{15}$
			\item $\frac{7}{15} - \frac{11}{4}$
			\item $\frac{11}{4} . \frac{14}{2}$
			\item $\frac{14}{2} : \frac{14}{12}$
			\item $\left(\frac{14}{12}\right) ^ 3$
		\end{enumerate}

	\item Observe as frações em cada item abaixo e assinale $<$, $>$ ou $=$. \textbf{(1.5 pnts)}

		\begin{enumerate}
			\item $\frac{13}{11} e \frac{12}{6}$
			\item $\frac{12}{6} e \frac{1}{12}$
			\item $\frac{1}{12} e \frac{2}{10}$
			\item $\frac{2}{10} e \frac{9}{14}$
			\item $\frac{9}{14} e \frac{10}{7}$
			\item $\frac{10}{7} e \frac{14}{11}$
		\end{enumerate}

	\item Encontre a fração geratriz para cada dízima periódica abaixo. \textbf{(2.0 pnts)}

		\begin{enumerate}
			\item $\frac{2}{4}$
			\item $\frac{1}{13}$
			\item $\frac{15}{6}$
			\item $\frac{4}{15}$
		\end{enumerate}

	\item Escreva por extenso o nome de cada fração abaixo. \textbf{(1.0 pnts)}

		\begin{enumerate}
			\item $\frac{12}{13}$
			\item $\frac{14}{2}$
			\item $\frac{1}{11}$
			\item $\frac{3}{14}$
		\end{enumerate}

	\item Calcule as potências a seguir. \textbf{(1.5 pnts)}

		\begin{enumerate}
			\item $\frac{15}{9}$
			\item $\frac{11}{2}$
			\item $\frac{8}{10}$
			\item $\frac{7}{2}$
			\item $\frac{1}{10}$
		\end{enumerate}

	\item Para cada fração a seguir, faça uma representação geométrica. \textbf{(1.0 pnts)}

		\begin{enumerate}
			\item $\frac{10}{3}$
			\item $\frac{15}{4}$
			\item $\frac{11}{13}$
			\item $\frac{9}{2}$
			\item $\frac{1}{11}$
			\item $\frac{10}{9}$
		\end{enumerate}

\end{enumerate}
\end{document}