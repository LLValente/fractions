\documentclass{article}

\usepackage{graphicx}
\usepackage[brazilian]{babel}

\title{Avaliação}
\author{Lucas L. Valente}        
\date{2023-06-09}

\begin{document}
\maketitle
\newpage

\begin{enumerate}
	\item Calcule. \textbf{(2.0 pnts)}

		\begin{enumerate}
			\item $\frac{11}{3}$ + $\frac{6}{1}$
			\item $\frac{6}{1}$ - $\frac{10}{1}$
			\item $\frac{10}{1}$ . $\frac{13}{10}$
			\item $\frac{13}{10}$ : $\frac{2}{3}$
			\item $\left(\frac{2}{3}\right) ^ 2$
		\end{enumerate}

	\item Observe as frações em cada item abaixo e assinale $<$, $>$ ou $=$. \textbf{(1.5 pnts)}

		\begin{enumerate}
			\item $\frac{15}{8}$ e $\frac{13}{3}$
			\item $\frac{13}{3}$ e $\frac{10}{11}$
			\item $\frac{10}{11}$ e $\frac{7}{3}$
			\item $\frac{7}{3}$ e $\frac{3}{7}$
			\item $\frac{3}{7}$ e $\frac{1}{2}$
			\item $\frac{1}{2}$ e $\frac{5}{14}$
		\end{enumerate}

	\item Encontre a fração geratriz para cada dízima periódica abaixo. \textbf{(2.0 pnts)}

		\begin{enumerate}
			\item $\frac{15}{9}$
			\item $\frac{11}{12}$
			\item $\frac{15}{5}$
			\item $\frac{4}{11}$
		\end{enumerate}

	\item Escreva por extenso o nome de cada fração abaixo. \textbf{(1.0 pnts)}

		\begin{enumerate}
			\item $\frac{9}{10}$
			\item $\frac{2}{2}$
			\item $\frac{13}{9}$
			\item $\frac{1}{3}$
		\end{enumerate}

	\item Calcule as potências a seguir. \textbf{(1.5 pnts)}

		\begin{enumerate}
			\item $\frac{9}{3}$
			\item $\frac{11}{5}$
			\item $\frac{4}{15}$
			\item $\frac{1}{15}$
			\item $\frac{11}{6}$
		\end{enumerate}

	\item Para cada fração a seguir, faça uma representação geométrica. \textbf{(1.0 pnts)}

		\begin{enumerate}
			\item $\frac{8}{1}$
			\item $\frac{7}{8}$
			\item $\frac{11}{8}$
			\item $\frac{8}{7}$
			\item $\frac{1}{6}$
			\item $\frac{5}{9}$
		\end{enumerate}

\end{enumerate}
\end{document}