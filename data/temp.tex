\documentclass{article}

\usepackage{graphicx}
\usepackage[brazilian]{babel}

\title{Avaliação}
\author{Lucas L. Valente}        
\date{2023-06-09}

\begin{document}
\maketitle
\newpage

\begin{enumerate}
	\item Calcule. \textbf{(2.0 pnts)}

		\begin{enumerate}
			\item $\frac{14}{14}$ + $\frac{9}{1}$
			\item $\frac{9}{1}$ - $\frac{14}{6}$
			\item $\frac{14}{6}$ . $\frac{5}{3}$
			\item $\frac{5}{3}$ : $\frac{6}{10}$
			\item $\left(\frac{6}{10}\right) ^ 3$
		\end{enumerate}

	\item Observe as frações em cada item abaixo e assinale $<$, $>$ ou $=$. \textbf{(1.5 pnts)}

		\begin{enumerate}
			\item $\frac{4}{14}$ e $\frac{7}{5}$
			\item $\frac{7}{5}$ e $\frac{14}{12}$
			\item $\frac{14}{12}$ e $\frac{3}{13}$
			\item $\frac{3}{13}$ e $\frac{2}{13}$
			\item $\frac{2}{13}$ e $\frac{13}{3}$
			\item $\frac{13}{3}$ e $\frac{6}{13}$
		\end{enumerate}

	\item Encontre a fração geratriz para cada dízima periódica abaixo. \textbf{(2.0 pnts)}

		\begin{enumerate}
			\item $\frac{2}{4}$
			\item $\frac{2}{1}$
			\item $\frac{12}{10}$
			\item $\frac{12}{2}$
		\end{enumerate}

	\item Escreva por extenso o nome de cada fração abaixo. \textbf{(1.0 pnts)}

		\begin{enumerate}
			\item $\frac{11}{6}$
			\item $\frac{10}{3}$
			\item $\frac{12}{1}$
			\item $\frac{8}{9}$
		\end{enumerate}

	\item Calcule as potências a seguir. \textbf{(1.5 pnts)}

		\begin{enumerate}
			\item $\frac{5}{6}$
			\item $\frac{10}{6}$
			\item $\frac{3}{7}$
			\item $\frac{6}{9}$
			\item $\frac{7}{15}$
		\end{enumerate}

	\item Para cada fração a seguir, faça uma representação geométrica. \textbf{(1.0 pnts)}

		\begin{enumerate}
			\item $\frac{14}{12}$
			\item $\frac{10}{1}$
			\item $\frac{12}{3}$
			\item $\frac{15}{6}$
			\item $\frac{2}{2}$
			\item $\frac{3}{5}$
		\end{enumerate}

\end{enumerate}
\end{document}